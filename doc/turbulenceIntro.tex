%
%$Id: turbulenceIntro.tex,v 1.1 2003-03-10 09:42:25 gotm Exp $
%

\section{The turbulence model \label{sec:turbulenceIntro}}

\subsection{Introduction}
To close the differential equations for momentum and heat (or
buoyancy), a description of the turbulent fluxes of momentum, $\mean{u'
w'}$, and heat, $\mean{w' \theta'}$, are required\footnote{Here, a
prime denotes the fluctuating part of a turbulent quantity and
$\mean{\cdots}$ is the ensemble average.}. For brevity, we discuss
here only the case of stratification caused by temperature. Other
effects like salinity have also been included in GOTM in a
completely analogous manner (see \sect{sec:stratification}).


\subsubsection{Physics}

There are different levels of one-point closure models available in
GOTM to compute the vertical turbulent fluxes. Presently, all of them
rely on the idea, that theses fluxes can be computed as the product of
a so-called turbulent diffusivity and a mean flow gradient. Using this
assumption, the fluxes of momentum and heat can be expressed as
\begin{equation}
  \label{fluxes}
  \mean{u' w'}      = - \nu_t  \partder{u}{z} \comma
  \mean{v' w'}      = - \nu_t  \partder{v}{z} \comma
  \mean{w' \theta'} = - \nu'_t \partder{\theta}{z}
\point
\end{equation}
Using an analogy to the kinetic theory of gases, the vertical
turbulent diffusivities, $\nu_t$ and $\nu'_t$, are assumed to be the
product of a typical velocity scale of turbulence, $q$, times a
typical length scale, $l$, see \cite{TennekesLumley72}. The velocity
scale $q$ can e.g.\ be identified with the average value of the
turbulent fluctuations expressed by the turbulent kinetic energy, $k=
q^2 /2$.  Then, the diffusivities of momentum and heat can be written
as
\begin{equation}
  \label{nu}
  \nu_t = c_\mu k^\frac{1}{2} l  \comma   \nu'_t = {c_\mu}' k^\frac{1}{2} l  
  \comma
\end{equation}
where the dimensionless quantities $c_\mu$ and ${c_\mu}'$ are usually
referred to as the `stability functions'.  Depending on the level of
turbulent closure, these stability functions can be either constants
or functions of some non-dimensional flow parameters (see below).

There are different possibilities in GOTM to compute the scales $q$ (or
$k$) and $l$ appearing in \eq{nu}. According to the level of
complexity, they are ordered in GOTM in the following fashion.

\begin{enumerate}
\item Both, $k$ and $l$ are computed from algebraic relations. The
 algebraic equation for $k$ is based on a strongly simplified form of
 the transport equation of the turbulent kinetic energy. In contrast,
 the equation for the length-scale is fully empirical. This level of
 closure can be identified with the `level 2' model of
 \cite{MellorYamada82}, who also give further explanations. This model
 often is an over-simplification in many situations and can also lead to
 numerical instabilities.
\item At the next level, $k$ is computed from the differential
 transport equation for the turbulent kinetic energy. As
 before, the length-scale is computed from an empirical
 relation. Models of this type are quite popular in geophysical
 modelling. A description is given in \sect{sec:algebraiclength}.
\item In the so-called two-equation models, both, $k$ and $l$, are computed from
 differential transport equations. As before, $k$ follows from the
 transport equation of the turbulent kinetic energy. Now, however, also
 the length-scale is determined from a differential transport
 equation. This equation is usually not directly formulated for the
 length-scale, but for a related, length-scale determining
 variable. Presently, there are different possibilities for the
 length-scale determining variables implemented in GOTM, such as the
 rate of dissipation, $\epsilon$, or the product $kl$. They are
 discussed in \sect{sec:updateLength}.

 The main advantage of the two-equation models is their greater
 generality. There are, for example, a number of fundamental flows
 which cannot be reproduced with an algebraically prescribed
 length-scale. Examples are the temporal decay of homogeneous
 turbulence, the exponential decay of turbulence in stratified
 homogeneous shear flows, and the spatial decay of shear-free
 turbulence from a planar source. A discussion of these points is
 given in \sect{sec:generate} and \sect{sec:analyse}. Also see
 \cite{Umlaufetal2003} and \cite{UmlaufBurchard2003}.
\end{enumerate}

In addition to the hierarchy of turbulence models in terms of their
method to describe the length scale, GOTM also supports an ordering
scheme according to the extent to which transport equations for the
second moments of turbulent quantities are solved.
\begin{enumerate}
\item At the lowest level of this scheme, it is postulated that
 $c_\mu=c_\mu^0$ and $c'_\mu=c'^0_\mu$ are constant. Because these models also
 assume an isotropic tensor relation between the velocity gradient and
 the tensor of the Reynolds-stresses, they usually fail in situations
 of strong anisotropy, most notably in stably stratified, curved or
 shallow flows. In unstratified flows with balanced aspect ratios
 (which seldom occur in nature), however, they usually yield
 reasonable results. Models of this type are referred to as the
 `standard' models in the following.
\item Some problems associated with standard versions of the models can
 be ameliorated by making $c_\mu$ and $c'_\mu$ empirical functions of
 one or several significant non-dimensional flow parameters. At this
 level, the simplest approach would be to formulate empirical
 relations suggested from observations in the field or in the
 laboratory. An example of such a relation is the model of
 \cite{SchumannGerz95} which has been implemented in GOTM (see
 \sect{sec:sg}).
\item Another, more consistent, approach results from the solution of 
 simplified forms of the transport equations for the Reynolds-stresses
 and the turbulent heat fluxes in addition to the transport equations
 for $k$ and the length-scale determining variable.  Surprisingly, it
 turns out that under some assumptions, and after tedious algebra, the
 turbulent fluxes computed by these models can be expressed by
 \eq{nu}. The important difference is, however, that the existence of the eddy
 diffusivities is not a postulate, but a consequence of the model. The
 stability functions $c_\mu$ and $c'_\mu$ can be shown to become
 functions of some non-dimensional numbers like 
 \begin{equation}
   \label{alphaMN} 
    \alpha_M = \frac{k^2}{\epsilon^2} M^2 \comma 
    \alpha_N = \frac{k^2}{\epsilon^2} N^2 \comma 
 \end{equation} 
 with the shear-frequency, $M$, and the buoyancy frequency, $N$,
 computed as described in \sect{sec:uequation} and 
 \sect{sec:stratification}, respectively.

 Most of the known results for this type of models have been
 implemented into GOTM. An up-to-date account of their derivation can
 be found in \cite{Canutoetal2001a}. Their evaluation for the oceanic
 mixed layer has been extensively discussed by
 \cite{BurchardBolding2001}.
\end{enumerate}

Evidently, this short introduction cannot serve as an
introductory text on one-point turbulence modelling. It serves merely
as a place to define the most important quantities and relations used
in this manual. Readers not familiar with this subject will certainly
feel the need for a more in-depth discussion. An excellent introduction
to turbulence is still the book of \cite{TennekesLumley72}. A modern
and detailed approach to one and two-equation models for unstratified
flows is given in the book of \cite{Wilcox98}, and the effects of
stratification are discussed e.g.\ by \cite{Rodi87} and by
\cite{Burchard2002a}.



\subsubsection{Numerics}

The numerical approximation of the turbulence equations is in principle
carried out as explained in section \ref{SectionNumericsMean}. 
One basic difference is however due to the fact that turbulent
quantities are generally non-negative such that
it is necessary that the discretised forms of the physical
equations retain the principle of non-negativity. 
A typical model problem would be the following:

\begin{equation}
 \label{eq:burchard11}
 \partder{X}{t} =P-QX, \quad P,Q > 0
\end{equation}
with $X$ denoting any non-negative quantity, $P$ a non-negative
source term, $QX$ a non-negative linear sink term, and $t$ denoting
time. $P$ and $Q$ may depend on $X$ and $t$. It can easily be shown
that with \eq{eq:burchard11}, $X$ remains non-negative for any
non-negative initial value $X_0$ and limited $Q$.  For the $q^2
l$-equation and the $\epsilon$-equation (described in
\sect{sec:lengthscaleeq} and \sect{sec:dissipationeq}), 
$Q$ would be proportional to $q/l$ and $\epsilon
/k�$, repsectively.

A straight-forward, explicit discretisation in time of \eq{eq:burchard11}
can be written as
\begin{equation}
  \label{eq:burchard12}
  \frac{X^{n+1}-X^n}{\Delta t}=P^n-Q^nX^n
\end{equation}
with the superscripts denoting the old ($n$) and the new ($n+1$) time
level and $\Delta t$ denoting the time step.  In this case, the
numerical solution on the new time level would be
\begin{equation}
  \label{eq:burchard13}
  X^{n+1}_i= X^n_i(1-\Delta tQ^n_i)+\Delta t P_i^n
  \comma
\end{equation}
which is negative for negative right hand side of \eq{eq:burchard12},
provided that
\begin{equation}
  \label{eq:burchard14}
  \Delta t > \frac{X^n}{X^nQ^n-P^n}
  \point
\end{equation}

Since it is computationally unreasonable to restrict the time step in
such a way that \eq{eq:burchard14} is avoided, a numerical procedure
first published by \cite{Patankar80} is generally applied
\begin{equation}
  \label{eq:burchard15}
  \frac{X^{n+1}-X^n}{\Delta t}=P^n-Q^nX^{n+1}
  \comma
\end{equation}
which yields an always non-negative solution for $X^{n+1}$,
\begin{equation}
  \label{eq:burchard16}
   X^{n+1}= \frac{X^n+\Delta t P^n}{1+\Delta t Q^n}
   \point
\end{equation}
Thus, the so-called quasi-implicit formulation
\eq{eq:burchard15} by \cite{Patankar80}
is a sufficient condition for positivity applied in almost all
numerical turbulence models.

 
